% !TeX spellcheck = en_GB
\documentclass[]{scrartcl}
\usepackage{phd-thesis-proposal}
% \usepackage{biblatex}

%opening
\title{
    Neuro-symbolic Techniques for Intelligent Systems:
    Knowledge-driven Solutions for Next Generation AI
}
\subtitle{Thesis proposal, Computer Science and Engineering Ph.D. course, cycle 38}
\author{Matteo Magnini}
\date{A.A. 2023/2024}

\begin{document}
    
    \maketitle
    
    \begin{abstract}
        %
        This research delves into the cutting-edge domain of neuro-symbolic (NeSy) techniques within the realm of artificial intelligence (AI).
        %
        Focused on advancing intelligent systems, the study explores the integration of symbolic knowledge through innovative approaches such as symbolic knowledge extraction (SKE) and symbolic knowledge injection (SKI).
        %
        A comprehensive literature review establishes the foundation, revealing insights into the extraction of human-understandable symbolic knowledge from sub-symbolic predictors and its subsequent injection to guide the learning process of neural networks.
        %
        The thesis presents novel NeSy techniques including SKI, SKE and the transfer of symbolic knowledge between different domains.
        %
        Beyond theoretical contributions, practical implementations showcase the real-world applications of neuro-symbolic techniques.
        %
        Additionally, the research contributes to the design and implementation of AI-based software systems, multi-agent explanations, and explores the potential of large language models for real-world applications (e.g., healthcare, nutrition).
        %
        The study extends into the realm of fairness-aware AI systems, addressing intersectional fairness.
        %
        The thesis concludes with a discussion of the future of AI and the potential of neuro-symbolic techniques to advance intelligent systems.
        %
    \end{abstract}
    
    \section{What has been done}
    \label{sec:done}

    During the first year of my Ph.D., I have studied the scientific literature on neuro-symbolic (NeSy) artificial intelligence (AI) approaches.
    %
    Because NeSy is a broad research area, my focus has been on the sub-fields of symbolic knowledge extraction (SKE) and symbolic knowledge injection (SKI).
    %
    SKE consists of \emph{extracting} symbolic knowledge (i.e., human understandable such as logic rules) from sub-symbolic predictors (e.g., neural networks).
    %
    SKI consists of \emph{injecting} symbolic knowledge into sub-symbolic predictors in order to guide their learning process (i.e., be compliant with the knowledge) and ultimately improve their performance.
    %
    The study of a huge amount of articles about SKE and SKI will result in a scientific literature review paper on NeSy approaches for AI (accepted on ACM transactions, will soon be published).

    After that, I have developed a couple of SKI approaches supporting the injection of stratified Datalog with negation into neural networks that use different strategies (KINS~\cite{kins-cilc2022} and KILL~\cite{kill-woa2022}).
    %
    Both algorithms have been implemented in Python and available on GitHub\footnote{\label{foot:psyki}\url{https://github.com/psykei/psyki-python}}.
    %

    At the same time, I have developed a platform for symbolic knowledge injection into sub-symbolic predictors (PSyKI)~\cite{psyki-extraamas2022}$^{\ref{foot:psyki}}$.
    %
    PSyKI is a Python library that allows the injection of symbolic knowledge into neural networks.
    %
    Currently, PSyKI supports 3 SKI algorithms (KBANN~\cite{TowellAaai1990}, KINS, and KILL) and a set of quality of service (QoS) metrics to evaluate the performance of the algorithms~\cite{skiqos-jaamas37}.
    %
    Along with this library, I have also implemented PSyKE~\cite{psyke-woa2021} in Python, available on GitHub\footnote{\label{foot:psyke}\url{https://github.com/psykei/psyke-python}}.

    Then, I have studied, designed and implemented AI based software systems that exploit knowledge injection / extraction techniques.
    %
    The first one is a recommender system for nutritional recipes that exploits symbolic knowledge extraction from neural networks in order to recommend recipes following both user's preferences and nutritional constraints~\cite{skerecommender-cmbp235}.
    %
    Then, I have co-written a protocol for multi-agent based explanations~\cite{explanationprotocol-extraamas2023}.
    %
    Also, I have explored the recent advances in large language models (LLMs) and their potential for AI applications (one paper submitted, one paper in preparation).
    %
    The first work consists of a comparative analysis of LLMs for healthcare chatbots in the context of hypertension.
    %
    The second work consists of a study on how to exploit LLMs as oracles for instantiating ontologies with domain-specific knowledge.

    Finally, I have been involved in the development of a framework for fairness-aware AI systems.
    %
    Currently I have two papers under review about this topic, one of this takles the problem of intersectional fairness.

    Along with the research activities, I have also been involved in teaching activities, Ph.D. courses, conferences and one summer school.


    \subsection{Scientific publications}
    \label{sec:done:publications}
    %
    During the first year of my Ph.D., I have done several activities including scientific publications, developing research technologies, and teaching.
    %
    Concerning my scientific articles, this is the list of the papers I have published so far:
    %
    \begin{itemize}
        \item ``On the Design of PSyKI: a Platform for Symbolic Knowledge Injection into Sub-Symbolic Predictors''~\cite{psyki-extraamas2022}
        \item ``KINS: Knowledge Injection via Network Structuring''~\cite{kins-cilc2022}
        \item ``A view to a KILL: Knowledge Injection via Lambda Layer''~\cite{kill-woa2022}
        \item ``Bridging Symbolic and Sub-Symbolic AI: Towards Cooperative Transfer Learning in Multi-Agent Systems''~\cite{ctl-aixia2022}
        \item ``Symbolic Knowledge Extraction for Explainable Nutritional Recommenders''~\cite{skerecommender-cmbp235}
        \item ``Knowledge injection of Datalog rules via Neural Network Structuring with KINS''~\cite{kins-jlc2023}
        \item ``Symbolic Knowledge Injection meets Intelligent Agents: QoS metrics and experiments''~\cite{skiqos-jaamas37}
        \item ``A General-Purpose Protocol for Multi-Agent based Explanations''~\cite{explanationprotocol-extraamas2023}
    \end{itemize}
    %
    Already accepted papers waiting for publication:
    %
    \begin{itemize}
        \item ``Symbolic Knowledge Extraction and Injection with Sub-symbolic Predictors: a Systematic Literature Review'', ACM Computing Surveys (CSUR).
        \item ``LLM-based Solutions for Healthcare Chatbots: a Comparative Analysis'', 3rd International Workshop on Telemedicine and E-Health Evolution in the New Era Of Social Distancing (TELMED 2024);
    \end{itemize}
    %
    I have also submitted / will submit the following papers:
    %
    \begin{itemize}
        \item ``Addressing Intersectional Fairness: a Practical Approach with FaUCI'', submitted at the 33rd International Joint Conference on Artificial Intelligence (IJCAI 2024);
        \item ``Analysing Fairness in Neural Networks under Constraint Injection with FaUCI'' submitted at the 33rd International Joint Conference on Artificial Intelligence (IJCAI 2024);
        \item ``Are Symbolic Knowledge Injection Techniques Robust Against Data Quality Degradation?''submitted at the 40th Conference on Uncertainty in Artificial Intelligence (UAI 2024);
        \item ``Privacy-Preserving LLM-Based Chatbots for Chronic Disease Self-management'' will be submitted soon probably at the Journal of Artificial Intelligence in Medicine or at the International Journal of Medical Informatics;
        \item ``Large language models as oracles for instantiating ontologies with domain-specific knowledge'' will be submitted soon probably at the Journal of Web Semantics or at the Knowledge-Based Systems journal.
    \end{itemize}

    \subsection{Research technologies}
    \label{sec:done:technologies}
    %
    I have developed and I am currently maintaining the following research technologies:
    %
    \begin{itemize}
        \item \textbf{PSyKI}~\cite{psyki-extraamas2022}: a platform for symbolic knowledge injection into sub-symbolic predictors available at~$^{\ref{foot:psyki}}$;
        \item \textbf{PSyKE}~\cite{psyke-woa2021}: a knowledge injection technique based on neural network structuring available at~$^{\ref{foot:psyke}}$.
    \end{itemize}

    \subsection{Teaching}
    \label{sec:done:teaching}
    %
    I have been teaching the following courses:
    %
    \begin{itemize}
        \item teaching assistant for the curse ``Distributed System'' (A.Y. 2023/2024) for the Master's Degree in Computer Science and Engineering (24 hours);
        \item teacher at professional education course (May 2023) for BPER Data Analytics, BBS (8 hours);
        \item teaching assistant for the curse ``Fondamenti di Informatica'' (A.Y. 2022/2023) for the Bachelor's Degree in Electronics Engineering and Biomedical Engineering (35 hours);
        \item teaching assistant for the curse ``Sistemi Distribuiti'' (A.Y. 2022/2023) for the Master's Degree in Computer Science and Engineering (24 hours).
    \end{itemize}

    \subsection{Ph.D. courses}
    \label{sec:done:courses}
    %
    I have attended the following Ph.D. courses:
    %
    \begin{itemize}
        \item ``Comparative Introduction to Deep Learning Frameworks: TensorFlow, PyTorch and Jax'' (Nov. 2022, 1 CFU);
        \item ``Low-rank Approaches for Data Analysis: Models, Numerical Methods and Applications'' (Jan. 2023, 2 CFU);
        \item ``How to Write and Publish a Research Paper in Computer Science and Engineering'' ( May 2023, 2 CFU);
        \item ``Devops meets scientific research'' (July 2023, 4 CFU);
        \item ``Introduction to RL'' (Oct. 2023, 3 CFU).
    \end{itemize}

    \subsection{Conferences}
    \label{sec:done:conferences}
    %
    I have physically attended the following conferences:
    %
    \begin{itemize}
        \item \textbf{AIxIA 2022}: 21st International Conference of the Italian Association for Artificial Intelligence (Nov. 2022, Udine, Italy);
        \item \textbf{KR 2023}: 20th International Conference on Principles of Knowledge Representation and Reasoning (Sep. 2023, Rhodes, Greece).
    \end{itemize}

    \subsection{Summer schools}
    \label{sec:done:summerschools}
    %
    I have physically attended the following summer schools:
    %
    \begin{itemize}
        \item \textbf{DeepLearn2023} 10th International Gran Canaria School on Deep Learning (July 2023, Gran Canaria, Spain, 4 CFU).
    \end{itemize}

    \section{Plan for the remaining years}
    \label{sec:todo}
    %
    From March to June 2024, I will spend a period abroad at the University of Oslo under the supervision of Prof. Ana Ozaki.
    %
    During this period, I will work on neuro-symbolic artificial intelligence in the context of the \href{https://www.integreat.no/index.html}{Integreat} project.
    %
    Integreat will make machine learning more sustainable, accurate, trustworthy, and ethical.
    %
    It develops theories, methods, models and algorithms that integrate general and domain-specific knowledge with data.
    %    
    More in detail, the main activity during my period abroad will be working on the topic ``Ontology Learning from Large Language Models''.
    %
    The objective is the study, design and implementation of a software system that can learn ontologies from large language models (LLM).
    %
    The hypothesis to be tested is that LLM are well-suited to serve as oracles in a querying phase to build ontologies.
    %
    % In this way, the ontology is automatically filled with domain-specific knowledge.
    %
    As result, the ontology is quickly and automatically enriched with manifold instances, which experts may consider keeping, adjust, discard, or complement according to their own needs and expertise.
    %
    The design of the system will need to consider all relevant aspects of both ontologies and LLM.
    %
    Concerning LLM it is important to consider the potential for hallucinations, speed in proving responses, potential usage costs, etc.
    %
    Multiple models and diverse ontologies will be employed.
    %
    Each experiment will involve an evaluation phase using pre-defined metrics and well-known public ontologies as benchmark.

    During the remaining months, I plan to work on not yet sufficiently explored NeSy techniques.
    %
    In particular:
    %
    \begin{inlinelist}
        \item investigate methods to handle different type of data other than tabular data (e.g., images, text, etc.);
        %
        \item experiment how symbolic knowledge could be transferred from one domain / system to another;
        %
        \item organise all the work done so far in the final thesis.
    \end{inlinelist}


    
    %\bibliographystyle{my-bib-style}
    \bibliographystyle{plain}
    \bibliography{publications}
    
    
\end{document}